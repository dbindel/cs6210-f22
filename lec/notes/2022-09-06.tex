\documentclass[12pt, leqno]{article} %% use to set typesize
\usepackage{fancyhdr}
\usepackage[letterpaper=true,colorlinks=true,linkcolor=black]{hyperref}

\usepackage{amsfonts}
\usepackage{amsmath,amssymb,amsthm}
\usepackage{color}
\usepackage{tikz}
\usepackage{pgfplots}
\usepackage{listings}
%\usepackage{courier}
%\usepackage[utf8]{inputenc}
%\usepackage[russian]{babel}

\lstdefinelanguage{Julia}%
  {morekeywords={abstract,break,case,catch,const,continue,do,else,elseif,%
      end,export,false,for,function,immutable,import,importall,if,in,%
      macro,module,otherwise,quote,return,switch,true,try,type,typealias,%
      using,while},%
   sensitive=true,%
   alsoother={$},%
   morecomment=[l]\#,%
   morecomment=[n]{\#=}{=\#},%
   morestring=[s]{"}{"},%
   morestring=[m]{'}{'},%
}[keywords,comments,strings]%

\lstset{
  numbers=left,
  basicstyle=\ttfamily\footnotesize,
  numberstyle=\tiny\color{gray},
  stepnumber=1,
  numbersep=10pt,
}

\newcommand{\iu}{\ensuremath{\mathrm{i}}}
\newcommand{\bbR}{\mathbb{R}}
\newcommand{\bbC}{\mathbb{C}}
\newcommand{\calV}{\mathcal{V}}
\newcommand{\calE}{\mathcal{E}}
\newcommand{\calG}{\mathcal{G}}
\newcommand{\calW}{\mathcal{W}}
\newcommand{\calP}{\mathcal{P}}
\newcommand{\macheps}{\epsilon_{\mathrm{mach}}}
\newcommand{\matlab}{\textsc{Matlab}}
\newcommand{\uQ}{\underline{Q}}
\newcommand{\uR}{\underline{R}}

\newcommand{\ddiag}{\operatorname{diag}}
\newcommand{\fl}{\operatorname{fl}}
\newcommand{\nnz}{\operatorname{nnz}}
\newcommand{\tr}{\operatorname{tr}}
\renewcommand{\vec}{\operatorname{vec}}

\newcommand{\vertiii}[1]{{\left\vert\kern-0.25ex\left\vert\kern-0.25ex\left\vert #1
    \right\vert\kern-0.25ex\right\vert\kern-0.25ex\right\vert}}
\newcommand{\ip}[2]{\langle #1, #2 \rangle}
\newcommand{\ipx}[2]{\left\langle #1, #2 \right\rangle}
\newcommand{\order}[1]{O( #1 )}

\newcommand{\kron}{\otimes}


\newcommand{\hdr}[1]{
  \pagestyle{fancy}
  \lhead{Bindel, Fall 2022}
  \rhead{Matrix Computations}
  \fancyfoot{}
  \begin{center}
    {\large{\bf #1}}
  \end{center}
  \lstset{language=Julia,columns=flexible}  
}


\begin{document}

\hdr{2022-09-06}

\section{Modeling floating point}

The fact that normal floating point results have a relative error
bounded by $\macheps$ gives us a useful {\em model} for reasoning about
floating point error.  We will refer to this as the ``$1 + \delta$''
model.  For example, suppose $x$ is an exactly-represented input to
the Julia statement
\begin{lstlisting}
    z = 1-x*x
\end{lstlisting}
We can reason about the error in the computed $\hat{z}$ as follows:
\begin{align*}
  t_1 &= \fl(x^2) = x^2 (1+\delta_1) \\
  t_2 &= 1-t_1 = (1-x^2)\left( 1 - \frac{\delta_1 x^2}{1-x^2} \right) \\
  \hat{z}
  &= \fl(1-t_1)
    = z \left( 1 - \frac{\delta_1 x^2}{1-x^2} \right)(1+\delta_2) \\
  & \approx z \left( 1 - \frac{\delta_1 x^2}{1-x^2} +\delta_2 \right),
\end{align*}
where $|\delta_1|, |\delta_2| \leq \macheps$.  As before, we throw
away the (tiny) term involving $\delta_1 \delta_2$.
Note that if $z$ is close to zero (i.e.~if there is {\em cancellation} in the
subtraction), then the model shows the result may have a
large relative error.

\subsection{First-order error analysis}

Analysis in the $1+\delta$ model quickly gets to be a sprawling mess
of Greek letters unless one is careful.  A standard trick to get
around this is to use {\em first-order} error analysis in which we
linearize all expressions involving roundoff errors.  In particular,
we frequently use the approximations
\begin{align*}
  (1+\delta_1)(1+\delta_2) & \approx 1+\delta_1 + \delta_2 \\
  1/(1+\delta) & \approx 1-\delta.
\end{align*}
In general, we will resort to first-order analysis without comment.
Those students who think this is a sneaky trick to get around our
lack of facility with algebra\footnote{%
Which it is.
}
may take comfort in the fact that if $|\delta_i| < \macheps$, then
in double precision
\[
  \left| \prod_{i=1}^n (1+\delta_i) \prod_{i=n+1}^N (1+\delta_i)^{-1} \right| < (1+1.03 N \macheps)
\]
for $N < 10^{14}$ (and a little further).

\subsection{Shortcomings of the model}

The $1+\delta$ model has two shortcomings.  First, it is only valid
for expressions that involve normalized numbers --- most notably,
gradual underflow breaks the model.  Second, the model is sometimes
pessimistic.  Certain operations, such as taking a difference between
two numbers within a factor of $2$ of each other, multiplying or
dividing by a factor of two\footnote{Assuming that the result
does not overflow or produce a subnormal.}, or multiplying two
single-precision numbers into a double-precision result,
are {\em exact} in floating point.  There are useful operations
such as simulating extended precision using ordinary floating point
that rely on these more detailed properties of the floating point system,
and cannot be analyzed using just the $1+\delta$ model.

\section{Finding and fixing floating point problems}

Floating point arithmetic is not the same as real arithmetic.  Even
simple properties like associativity or distributivity of addition and
multiplication only hold approximately.  Thus, some computations that
look fine in exact arithmetic can produce bad answers in floating
point.  What follows is a (very incomplete) list of some of the ways
in which programmers can go awry with careless floating point
programming.

\subsection{Cancellation}

If $\hat{x} = x(1+\delta_1)$ and $\hat{y} = y(1+\delta_2)$ are
floating point approximations to $x$ and $y$ that are very close, then
$\fl(\hat{x}-\hat{y})$ may be a poor approximation to $x-y$ due to
{\em cancellation}.  In some ways, the subtraction is blameless in
this tail: if $x$ and $y$ are close, then $\fl(\hat{x}-\hat{y}) =
\hat{x}-\hat{y}$, and the subtraction causes no additional rounding
error.  Rather, the problem is with the approximation error already
present in $\hat{x}$ and $\hat{y}$.

The standard example of loss of accuracy revealed through cancellation
is in the computation of the smaller root of a quadratic using the
quadratic formula, e.g.
\[
  x = 1-\sqrt{1-z}
\]
for $z$ small.  Fortunately, some algebraic manipulation gives an
equivalent formula that does not suffer cancellation:
\[
  x =
    \left( 1-\sqrt{1-z} \right)
    \left(\frac{1+\sqrt{1-z}}{1+\sqrt{1-z}}\right)
  =
    \frac{z}{1+\sqrt{1-z}}.
\]

\subsection{Sensitive subproblems}

We often solve problems by breaking them into simpler subproblems.
Unfortunately, it is easy to produce badly-conditioned subproblems
as steps to solving a well-conditioned problem.  As a simple (if
contrived) example, try running the following Julia code:

\begin{lstlisting}
function silly_sqrt(n=100)
  x = 2.0
  for k = 1:n
    x = sqrt(x)
  end
  for k = 1:n
    x = x^2
  end
  x
end
\end{lstlisting}

In exact arithmetic, this should produce 2, but what does it produce
in floating point?  In fact, the first loop produces a correctly
rounded result, but the second loop represents the function
$x^{2^{60}}$, which has a condition number far greater than $10^{16}$
--- and so all accuracy is lost.

\subsection{Unstable recurrences}

One of my favorite examples of this problem is the recurrence relation
for computing the integrals
\[
  E_n = \int_{0}^1 x^n e^{x-1} \, dx.
\]
Integration by parts yields the recurrence
\begin{align*}
  E_0 &= 1-1/e \\
  E_n &= 1-nE_{n-1}, \quad n \geq 1.
\end{align*}
This looks benign enough at first glance: no single step of
this recurrence causes the error to explode.  But
each step amplifies the error somewhat, resulting in an exponential
growth in error\footnote{%
Part of the reason that I like this example is that one can
run the recurrence {\em backward} to get very good results,
based on the estimate $E_n \approx 1/(n+1)$ for $n$ large.
}.

\subsection{Undetected underflow}

In Bayesian statistics, one sometimes computes ratios of long
products.  These products may underflow individually, even when the
final ratio is not far from one.  In the best case, the products will
grow so tiny that they underflow to zero, and the user may notice an
infinity or NaN in the final result.  In the worst case, the
underflowed results will produce nonzero subnormal numbers with
unexpectedly poor relative accuracy, and the final result will be
wildly inaccurate with no warning except for the (often ignored)
underflow flag.

\subsection{Bad branches}

A NaN result is often a blessing in disguise: if you see an
unexpected NaN, at least you {\em know} something has gone wrong!
But all comparisons involving NaN are false,  and so
when a floating point result is used to compute a branch condition
and an unexpected NaN appears, the result can wreak havoc.
As an example, try out the following code in Julia with `0.0/0.0` as input.

\begin{lstlisting}
function test_negative(x)
  if x < 0.0
    "$(x) is negative"
  elseif x >= 0.0
    "$(x) is non-negative"
  else
    "$(x) is ... uh..."
  end
end\end{lstlisting}

\section{Sums and dots}

We already described a couple of floating point examples that
involve evaluation of a fixed formula (e.g.~computation of
the roots of a quadratic).  We now turn to the analysis of
some of the building blocks for linear algebraic computations:
sums and dot products.

\subsection{Sums two ways}

As an example of first-order error analysis, consider the following
code to compute a sum of the entries of a vector $v$:
\begin{lstlisting}
  s = 0
  for k = 1:n
    s += v[k]
  end
\end{lstlisting}
Let $\hat{s}_k$ denote the computed sum at step $k$ of the loop;
then we have
\begin{align*}
  \hat{s}_1 &= v_1 \\
  \hat{s}_k &= (\hat{s}_{k-1} + v_k)(1 + \delta_k), \quad k > 1.
\end{align*}
Running this forward gives
\begin{align*}
  \hat{s}_2 &= (v_1 + v_2)(1+\delta_2) \\
  \hat{s}_3 &= ((v_1 + v_2)(1+\delta_2) + v_3)(1+\delta_2)
\end{align*}
and so on.  Using first-order analysis, we have
\[
  \hat{s}_k \approx (v_1 + v_2)\left(1 + \sum_{j=2}^k \delta_j \right)
              + \sum_{l=3}^k v_l \left( 1 + \sum_{j=l}^k \delta_j \right),
\]
and the difference between $\hat{s}_k$ and the exact partial sum
is then
\[
  \hat{s}_k-s_k \approx \sum_{j=2}^k s_j \delta_j.
\]
Using $\|v\|_1$ as a uniform bound on all the partial sums, we have
\[
  |\hat{s}_n-s_n| \lesssim (n-1) \macheps \|v\|_2.
\]

An alternate analysis, which is a useful prelude to analyses to come
involves writing an error recurrence.
Taking the difference between $\hat{s}_k$ and the true partial sums $s_k$,
we have
\begin{align*}
  e_1 &= 0 \\
  e_{k} &= \hat{s}_k-s_k \\
        &= (\hat{s}_{k-1} + v_k)(1+\delta_k) - (s_{k-1} + v_k) \\
        &= e_{k-1} + (\hat{s}_{k-1} + v_k) \delta_k,
\end{align*}
and $\hat{s}_{k-1} + v_k = s_k + O(\macheps)$, so that
\[
  |e_{k}| \leq |e_{k-1}| + |s_k| \macheps + O(\macheps^2).
\]
Therefore,
\[
  |e_{n}| \lesssim (n-1) \macheps \|v\|_1,
\]
which is the same bound we had before.

\subsection{Backward error analysis for sums}

In the previous subsection, we showed an error analysis for partial sums
leading to the expression:
\[
  \hat{s}_n \approx (v_1 + v_2) \left(1 + \sum_{j=2}^n \delta_j \right)
              + \sum_{l=3}^n v_l \left( 1 + \sum_{j=l}^n \delta_j \right).
\]
We then proceded to aggregate all the rounding error terms in order to
estimate the error overall.  As an alternative to aggregating the roundoff,
we can also treat the rounding errors as perturbations to the input
variables (the entries of $v$); that is, we write the computed sum as
\[
  \hat{s}_n = \sum_{j=1}^n \hat{v}_j
\]
where
\[
  \hat{v}_j = v_j (1+\eta_j), \quad
  \mbox{ where }|\eta_j| \lesssim (n+1-j) \macheps.
\]
This gives us a {\em backward error} formulation of the rounding:
we have re-cast the role of rounding error in terms of a
perturbation to the input vector $v$.  In terms of the 1-norm,
we have the relative error bound
\[
  \|\hat{v}-v\|_1 \lesssim n \macheps \|v\|_1;
\]
or we can replace $n$ with $n-1$ by being a little more careful.
Either way, what we have shown is that the summation algorithm
is {\em backward stable}, i.e.~we can ascribe the roundoff to
a (normwise) small relative error with a bound of $C \macheps$
where the constant $C$ depends on the size $n$ like some low-degree
polynomial.

Once we have a bound on the backward error, we can bound the forward
error via a condition number.  That is, suppose we write the true
and perturbed sums as
\begin{align*}
  s &= \sum_{j=1}^n v_j &
  \hat{s} &= \sum_{j=1}^n \hat{v}_j.
\end{align*}
We want to know the relative error in $\hat{s}$ via a normwise
relative error bound in $\hat{v}$, which we can write as
\[
  \frac{|\hat{s}-s|}{|s|} =
  \frac{|\sum_{j=1}^n (\hat{v}_j-v_j)|}{|s|} \leq
  \frac{\|\hat{v}-v\|_1}{|s|} =
  \frac{\|v\|_1}{|s|} \frac{\|\hat{v}-v\|_1}{\|v\|_1}.
\]
That is, $\|v\|_1/|s|$ is the condition number for the summation
problem, and our backward stability analysis implies
\[
  \frac{|\hat{s}-s|}{|s|} \leq
  \frac{\|v\|_1}{|s|} n \macheps.
\]
This is the general pattern we will see again in the future:
our analysis consists of a backward error computation that depends
purely on the algorithm, together with a condition number that depends
purely on the problem.  Together, these give us forward error bounds.

\subsection{Running error bounds for sums}

In all the analysis of summation we have done so far, we ultimately
simplified our formulas by bounding some quantity in terms of $\|v\|_1$.
This is nice for algebra, but we lose some precision in the process.
An alternative is to compute a {\em running error bound}, i.e.~augment
the original calculation with something that keeps track of the error
estimates.  We have already seen that the error in the computations
looks like
\[
  \hat{s}_n-s_n = \sum_{j=2}^n s_j \delta_j + O(\macheps^2),
\]
and since $s_j$ and $\hat{s}_j$ differ only by $O(\macheps)$ terms,
\[
  |\hat{s}_n-s_n| \lesssim \sum_{j=2}^n |\hat{s}_j| \macheps + O(\macheps^2),
\]
We are not worried about doing a rounding error analysis of our rounding
error analysis --- in general, we care more about order of magnitude for
rounding error anyhow --- so the following routine does an adequate job
of computing an (approximate) upper bound on the error in the summation:
\begin{lstlisting}
  s = 0.0
  e = 0.0
  for k = 1:n
    s += v[k]
    e += abs(s) * eps(Float64);
  end
\end{lstlisting}

\subsection{Compensated summation}

We conclude our discussion of rounding analysis for summation
with a comment on the {\em compensated summation} algorithm of
Kahan, which is not amenable to straightforward $1+\delta$ analysis.
The algorithm maintains the partial sums not as a single variable
{\tt s}, but as an unevaluated sum of two
variables {\tt s} and {\tt c}:
\begin{lstlisting}
  s = 0.0
  c = 0.0
  for k = 1:n
    y = v[i] - c
    t = s + y
    c = (t - s) - y  # Key step
    s = t
  end
\end{lstlisting}
Where the error bound for ordinary summation is
$(n-1) \macheps \|v\|_1 + O(\macheps^2)$,
the error bound for compensated summation is
$2 \macheps \|v\|_1 + O(\macheps^2)$.  Moreover,
compensated summation is exact for adding up to $2^k$
terms that are within about $2^{p-k}$ of each other
in magnitude.

Nor is Kahan's algorithm the end of the story!
Higham's {\em Accuracy and Stability of Numerical Methods}
devotes an entire chapter to summation methods, and there
continue to be papers written on the topic.  For our purposes,
though, we will wrap up here with two observations:
\begin{itemize}
\item Our initial analysis in the $1+\delta$ model illustrates the
  general shape these types of analyses take and how we can re-cast
  the effect of rounding errors as a ``backward error'' that perturbs
  the inputs to an exact problem.
\item The existence of algorithms like Kahan's compensated summation
  method should indicate that the backward-error-and-conditioning
  approach to rounding analysis is hardly the end of the story.
  One could argue it is hardly the beginning!  But it is the approach
  we will be using for most of the class.
\end{itemize}

\subsection{Dot products}

We now consider another example, this time involving
a real dot product computed by a loop of the form
\begin{lstlisting}
  dot = 0
  for k = 1:n
    dot += x[k]*y[k];
  end
\end{lstlisting}
Unlike the simple summation we analyzed above, the dot product involves
two different sources of rounding errors: one from the summation,
and one from the product.  As in the case of simple summations, it is
convenient to re-cast this error in terms of perturbations to the input.
We could do this all in one go, but since we have already spent so much
time on summation, let us instead do it in two steps.  Let $v_k = x_k y_k$;
in floating point, we get $\hat{v}_k = v_k (1+\eta_k)$
where $|\eta_k| < \macheps$.  Further,
we have already done a backward error analysis of summation to show
that the additional error in summation can be cast onto the summands,
i.e.~the floating point result is $\sum_k \tilde{v}_k$ where
\begin{align*}
  \tilde{v}_k
  &= \hat{v}_k (1 + \sum_{j=\min(2,n)}^n \delta_j) (1 + \eta_k) + O(\macheps^2) \\
  &= v_k (1+\gamma_k) + O(\macheps^2)
\end{align*}
where
\[
  |\gamma_k| = |\eta_k + \sum_{j=\min(2,n)}^n \delta_j| \leq n \macheps.
\]
Rewriting $v_k(1+\gamma_k)$ as $\hat{x}_k y_k$ where
$\hat{x}_k = x_k (1+\gamma_k)$, we have that the computed inner
product $y^T x$ is equivalent to the exact inner product of $y^T \hat{x}$
where $\hat{x}$ is an elementwise relatively accurate
(to within $n \macheps + O(\macheps^2)$) approximation to $x$.

A similar backward error analysis shows that computed matrix-matrix
products $AB$ in general can be interpreted as $\hat{A} B$ where
\[
  |\hat{A}-A| < p \macheps |A| + O(\macheps^2)
\]
and $p$ is the inner dimension of the product.  Exactly what $\hat{A}$
is depends not only on the data, but also the loop order used in the
multiply --- since, as we recall, the order of accumulation may vary
from machine to machine depending on what blocking is best suited to
the cache!  But the bound on the backward error holds for all the common
re-ordering\footnote{%
For those of you who know about Strassen's algorithm --- it's not
backward stable, alas.
}
And this backward error characterization, together with the
type of sensitivity analysis for matrix multiplication that we have
already discussed, gives us a uniform framework for obtaining
forward error bounds for matrix-matrix muliplication; and the same
type of analysis will continue to dominate our discussion of rounding
errors as we move on to more complicated matrix computations.

\subsection{Back-substitution}

We now consider the floating point analysis of a standard
{\em back-substitution} algorithm for solving an upper triangular
system
\[
  Uy = b.
\]
To solve such a linear system, we process each row in turn in reverse
order to find the value of the corresponding entry of $y$.  For
example, for the 3-by-3 case with
\[
U = \begin{bmatrix} 1 & 3 & 5 \\ & 4 & 2 \\ & & 6 \end{bmatrix}, \quad
b = \begin{bmatrix} 1 \\ -12 \\ 12 \end{bmatrix}
\]
Back substitution proceeds row-by-row:
\begin{itemize}
\item[Row 3:] $6 y_3 = 12$ (so $y_3 = 12/2 = 2$)
\item[Row 2:] $4 y_2 + 2 y_3 = -12$ (so $y_2 = (-12-2y_3)/4 = -4$)
\item[Row 1:] $y_1 + 3 y_2 + 5 y_3 = 1$ (so $y_1 = (1-3y_2-5y_3)/1 = 3$)
\end{itemize}
More generally, we have
\[
  y_i = \left( b_i - \sum_{j > i} u_{ij} y_j \right)/u_{ii}.
\]
In code, if we weren't inclined to just write \verb|y=U\b|, we might
write this as
\begin{lstlisting}
y = copy(b)
for i = n:-1:1
    # Loop equivalent to y[i] -= dot(U[i,i+1:end], y[i+1:end])
    for j = i+1:n
        y[i] -= U[i,j]*y[j]
    end
    y[i] /= U[i,i]
end
\end{lstlisting}

If we evaluate this in floating point arithmetic as a dot product,
subtraction, and division, we get that
\[
  \hat{y}_i = \left( b_i - \sum_{j > i} \hat{u}_{ij} \hat{y}_j
  \right)/u_{ii} \cdot (1+\delta_1) (1+\delta_2)
\]
where the $\hat{y}_j$ terms are the previously-computed entries in the
$y$ vector, the $\hat{u}_{ij}$ terms are the $u_{ij}$ with a $(n-i-1)
\macheps$ backward error modification from the dot product, the
$\delta_1$ error is associated with the subtraction and the $\delta_2$
error is associated with the division.  This in turn gives us that
\[
  \hat{y}_i = \left( b_i - \sum_{j > i} \hat{u}_{ij} \hat{y}_j \right)/\hat{u}_{ii}
\]
where
\[
  \hat{u}_{ii} = \frac{u_{ii}}{(1+\delta_1)(1+\delta_2)} =
  u_{ii} (1-\delta_1-\delta_2+ O(\macheps^2)).
\]
That is, we can recast the final subtraction and division
as a relative perturbation of $\lesssim 2 \macheps$ to the diagonal.
Putting everything together, we have that
\[
  \hat{U} \hat{y} = b
\]
where $|\hat{U}-U| \lesssim n \macheps |U|$.

\section{Error analysis for linear systems}

We now discuss the sensitivity of linear systems to perturbations.
This is relevant for two reasons:
\begin{enumerate}
\item Our standard recipe for getting an error bound for a computed
  solution in the presence of roundoff is to combine a backward
  error analysis (involving only features of the algorithm) with
  a sensitivity analysis (involving only features of the problem).
  We saw an example above: we know that the standard back-substitution
  process results in a backward error like $n \macheps |U|$, but what
  does that mean for solutions of the linear system?
\item Even without rounding error, it is important to understand
  the sensitivity of a problem to the input variables if the inputs
  are in any way inaccurate (e.g.~because they come from measurements).
\end{enumerate}
We describe several different bounds that are useful in different
contexts.

\subsection{First-order analysis}

We begin with a discussion of the first-order sensitivity analysis
of the system
\[
  Ax = b.
\]
Using our favored variational notation, we have the following relation
between perturbations to $A$ and $b$ and perturbations to $x$:
\[
  \delta A \, x + A \, \delta x = \delta b,
\]
or, assuming $A$ is invertible,
\[
  \delta x = A^{-1} (\delta b - \delta A \, x).
\]
We are interested in relative error, so we divide through by $\|x\|$:
\[
  \frac{\|\delta x\|}{\|x\|} \leq
  \frac{\|A^{-1} \delta b \|}{\|x\|} + \frac{\|A^{-1} \delta A \, x\|}{\|x\|}
\]
The first term is bounded by
\[
  \frac{\|A^{-1} \delta b \|}{\|x\|} \leq
  \frac{\|A^{-1}\| \|\delta b\|}{\|x\|} =
  \kappa(A) \frac{\|\delta b\|}{\|A\| \|x\|} \leq
  \kappa(A) \frac{\|\delta b\|}{\|b\|}
\]
and the second term is bounded by
\[
  \frac{\|A^{-1} \delta A \, x\|}{\|x\|} \leq
  \frac{\|A^{-1} \| \|\delta A\| \|x\|}{\|x\|} =
  \kappa(A) \frac{\|\delta A\|}{\|A\|}
\]
Putting everything together, we have
\[
  \frac{\|\delta x\|}{\|x\|} \leq\kappa(A)
    \left( \frac{\|\delta A\|}{\|A\|} + \frac{\|\delta b\|}{\|b\|} \right),
\]
That is, the relative error in $x$ is (to first order) bounded by the
condition number times the relative errors in $A$ and $b$.

\subsection{Beyond first order}

What if we want to go beyond the first-order error analysis?
Suppose that
\[
  Ax = b \quad \mbox{ and } \quad \hat{A} \hat{x} = \hat{b}.
\]
Then (analogous to our previous manipulations),
\[
  (\hat{A}-A) \hat{x} + A(\hat{x}-x) = \hat{b}-b
\]
from which we have
\[
  \hat{x}-x = A^{-1} \left( (\hat{b}-b) - E \hat{x} \right),
\]
where $E \equiv \hat{A}-A$.  Following the same algebra as before,
we have
\[
  \frac{\|\hat{x}-x\|}{\|x\|} \leq
  \kappa(A) \left(
    \frac{\|E\|}{\|A\|} \frac{\|\hat{x}\|}{\|x\|} +
    \frac{\|\hat{b}-b\|}{\|b\|}
  \right).
\]
Assuming $\|A^{-1}\| \|E\| < 1$, a little additional algebra
(left as an exercise to the student) yields
\[
\frac{\|\hat{x}-x\|}{\|x\|} \leq
\frac{\kappa(A)}{1-\|A^{-1}\|\|E\|} \left(
  \frac{\|E\|}{\|A\|} +
  \frac{\|\hat{b}-b\|}{\|b\|}
\right).
\]

Is this an important improvement on the first order bound?
Perhaps not, for two reasons:
\begin{itemize}
\item One typically cares about the order of magnitude of possible error,
  not the exact bound, and
\item The first-order bound and the ``true'' bound only disagree when
  both are probably pretty bad.  When our house is in flames, our first
  priority is not to gauge whether the garage will catch as well;
  rather, we want to call the firefighters to put it out!
\end{itemize}

\subsection{Componentwise relative bounds}

What if we have more control over the perturbations than a simple bound
on the norms?  For example, we might have a componentwise perturbation
bound
\[
  |\delta A| < \epsilon_A |A| \quad |\delta b| < \epsilon_b |b|,
\]
and neglecting $O(\epsilon^2)$ terms, we obtain
\[
  |\delta x|
  \leq |A^{-1}| \left( \epsilon_b |b| + \epsilon_A |A| |x| \right)
  \leq (\epsilon_b+\epsilon_A) |A^{-1}| |A| |x|.
\]
Taking any vector norm such that $\|\,|x|\,\|=\|x\|$, we have
\[
  \|\delta x\| \leq (\epsilon + \epsilon') \| \, |A^{-1}| \, |A| \, \|.
\]
The quantity
$\kappa_{\mathrm{rel}}(A) = \| \, |A^{-1}| \, |A| \, \|$ is the
componentwise relative condition number (also known as the Skeel
condition number).

\subsection{Residual-based bounds}

The {\em residual} for an approximate solution $\hat{x}$ to the
equation $Ax = b$ is
\[
  r = A \hat{x} - b.
\]
We can express much simpler error bounds in terms of the residual,
using the relation
\[
  \hat{x}-x = A^{-1} r;
\]
taking norms immediately gives
\[
  \|\hat{x}-x\| \leq \|A^{-1}\| \|r\|
\]
and for any vector norm such that $\|\,|x|\,\|=\|x\|$, we have
\[
  \|\hat{x}-x\| \leq \| \, |A^{-1}| |r| \, \|.
\]
Note that we can re-cast a residual error as a backward error on $A$
via the relation
\[
  \left( A - \frac{r\hat{x}^T}{\|\hat{x}\|^2} \right) \hat{x} = b.
\]

\subsection{Shape of error}

So far, we have only really discussed the {\em magnitude} of
errors in a linear solve, but it is worth taking a moment to
consider the {\em shape} of the errors as well.  In particular,
suppose that we want to solve $Ax = b$, and we have the
singular value decomposition
\[
  A = U \Sigma V^T.
\]
If $\sigma_n(A) \ll \sigma_1(A)$, then $\kappa_2 = \sigma_1/\sigma_n \gg q$,
and we expect a large error.  But is this the end of the story?
Suppose that $A$ satisfies
\[
  1 \geq \sigma_1 \geq \ldots \geq \sigma_k \geq C_1 ~~ > ~~
  C_2 \geq \sigma_{k+1} \geq \ldots \geq \sigma_n > 0.
\]
where $C_1 \gg C_2$.  Let $r = A \hat{x} - b$, so that $Ae = r$
where $e = \hat{x}-x$.  Then
\[
  e = A^{-1} r = V \Sigma^{-1} U^T r = V \Sigma^{-1} \tilde{r}
    = \sum_{j=1}^n \frac{\tilde{r}_j}{\sigma_j} v_j.
\]
where $\|\tilde{r}\| = \|U^T r\| = \|r\|$.  Split this as
\[
  e = e_1 + e_2
\]
where we have a controlled piece
\[
  \|e_1\| = \left\| \sum_{j=1}^k \frac{\tilde{r}_j}{\sigma_j} v_j \right\|
          \leq \frac{\|r\|}{C_1}
\]
and a piece that may be large,
\[
  e_2 = \sum_{j={k+1}}^n \frac{\tilde{r}_j}{\sigma_j} v_j.
\]
Hence, backward stability implies that the error consists of a small
part and a part that lies in the ``nearly-singular subspace'' for the
matrix.

\end{document}
