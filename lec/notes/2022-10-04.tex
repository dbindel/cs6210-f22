\documentclass[12pt, leqno]{article} %% use to set typesize
\usepackage{fancyhdr}
\usepackage[letterpaper=true,colorlinks=true,linkcolor=black]{hyperref}

\usepackage{amsfonts}
\usepackage{amsmath,amssymb,amsthm}
\usepackage{color}
\usepackage{tikz}
\usepackage{pgfplots}
\usepackage{listings}
%\usepackage{courier}
%\usepackage[utf8]{inputenc}
%\usepackage[russian]{babel}

\lstdefinelanguage{Julia}%
  {morekeywords={abstract,break,case,catch,const,continue,do,else,elseif,%
      end,export,false,for,function,immutable,import,importall,if,in,%
      macro,module,otherwise,quote,return,switch,true,try,type,typealias,%
      using,while},%
   sensitive=true,%
   alsoother={$},%
   morecomment=[l]\#,%
   morecomment=[n]{\#=}{=\#},%
   morestring=[s]{"}{"},%
   morestring=[m]{'}{'},%
}[keywords,comments,strings]%

\lstset{
  numbers=left,
  basicstyle=\ttfamily\footnotesize,
  numberstyle=\tiny\color{gray},
  stepnumber=1,
  numbersep=10pt,
}

\newcommand{\iu}{\ensuremath{\mathrm{i}}}
\newcommand{\bbR}{\mathbb{R}}
\newcommand{\bbC}{\mathbb{C}}
\newcommand{\calV}{\mathcal{V}}
\newcommand{\calE}{\mathcal{E}}
\newcommand{\calG}{\mathcal{G}}
\newcommand{\calW}{\mathcal{W}}
\newcommand{\calP}{\mathcal{P}}
\newcommand{\macheps}{\epsilon_{\mathrm{mach}}}
\newcommand{\matlab}{\textsc{Matlab}}
\newcommand{\uQ}{\underline{Q}}
\newcommand{\uR}{\underline{R}}

\newcommand{\ddiag}{\operatorname{diag}}
\newcommand{\fl}{\operatorname{fl}}
\newcommand{\nnz}{\operatorname{nnz}}
\newcommand{\tr}{\operatorname{tr}}
\renewcommand{\vec}{\operatorname{vec}}

\newcommand{\vertiii}[1]{{\left\vert\kern-0.25ex\left\vert\kern-0.25ex\left\vert #1
    \right\vert\kern-0.25ex\right\vert\kern-0.25ex\right\vert}}
\newcommand{\ip}[2]{\langle #1, #2 \rangle}
\newcommand{\ipx}[2]{\left\langle #1, #2 \right\rangle}
\newcommand{\order}[1]{O( #1 )}

\newcommand{\kron}{\otimes}


\newcommand{\hdr}[1]{
  \pagestyle{fancy}
  \lhead{Bindel, Fall 2022}
  \rhead{Matrix Computations}
  \fancyfoot{}
  \begin{center}
    {\large{\bf #1}}
  \end{center}
  \lstset{language=Julia,columns=flexible}  
}


\begin{document}

\hdr{2022-10-04}

\section{Choice of regularization}

All of the regularization methods we have discussed share a common
trait: they define a parametric family of models.  With more
regularization, we restrict the range of models we can easily generate
(adding bias), but we also reduce the sensitivity of the fit (reducing
variance). The choice of the regularization parameter is a key aspect of
these methods, and we now briefly discuss three different ways of
systematically making that choice.  In all cases, we
rely on the assumption that the sample observations we use for the
fit are representative of the population of observations where we might
want to predict.

\subsection{Morozov's discrepancy principle}

Suppose that we want to fit $Ax \approx \hat{b}$ by regularized least
squares, and the (noisy) observation vector $\hat{b}$ is known to be
within some error bound $\|e\|$ of the true values $b$. The discrepancy
principle says that we should choose the regularization parameter so the
residual norm is approximately $\|e\|$. That is, we seek the most stable
fitting problem we can get subject to the constraint that the residual
error for the regularized solution (with the noisy vector $\hat{b}$) is
not much bigger than we would get from unknown true solution.

One of the most obvious drawbacks of the discrepancy principle is that
it requires that we have an estimate for the norm of the error in the
data.  Sadly, such estimates are not always available.

\subsection{The L-curve}

A second approach to the regularization parameter is the {\em L-curve}.
If we draw a parametric curve of the residual error versus solution norm
on a log-log plot, with $\log \|r_{\lambda}\|$ on the $x$ axis
and $\log \|x_{\lambda}\|$ on the $y$ axis, we often see an ``L'' shape.
In the top of the vertical bar (small $\lambda$), we find that increasing
regularization decreases the solution norm significantly without significantly
increasing the residual error.  Along the end of the horizontal part,
increasing regularization increases the residual error, but does not
significantly help with the solution norm.  We want the corner of the
curve, where the regularization is chosen to minimize the norm of the
solution subject to the constraint that the residual is close to the
smallest possible residual (which we would have without regularization).

Computing the inflection point on the L-curve is a neat calculus
exercise which we will not attempt here.

\subsection{Cross-validation}

The idea with cross-validation is to choose the parameter
by fitting the model on a subset of the data and testing on the remaining
data.  We may do this with multiple partitions into data used for fitting
versus data reserved for checking predictions.
We often choose regularization parameters to give the smallest error on
the predictions in a cross-validation study.

One variant of cross-validation involves minimizing the
{\em leave-one-out cross-validation} (LOOCV) statistic:
\[
  \mbox{LOOCV} = \frac{1}{m} \sum_{i=1}^m \left[ Ax^{(-i)}-b \right]_i^2,
\]
where $x^{(-i)}$ denotes the model coefficients fit using all but the
$i$th data point.

To compute the LOOCV statistic in the most obvious way, we would
delete each row $a_i^T$ of $A$ in turn, fit the model coefficients
$x^{(-i)}$, and then evaluate $r^{(-i)} = b_i - a_i^T x^{(-i)}$.
This involves $m$
least squares problems, for a total cost of $O(m^2 n^2)$ (as opposed
to the usual $O(mn^2)$ cost for an ordinary least squares problem).
Let us find a better way!  For the sake of concreteness, we will focus
on the Tikhonov-regularized version of the problem

The key is to write the equations for $x^{(-i)}$ as a small change to
the equations for $(A^T A + \lambda^2 I) x^* = A^T b$:
\[
  (A^T A + \lambda^2 I - a_i a_i^T) x^{(-i)} = A^T b - a_i b_i.
\]
This subtracts the influence of row $i$ from both sides of the normal
equations.  By introducing the auxiliary variable $\gamma = -a_i^T x^{(-i)}$,
we have
\[
  \begin{bmatrix}
    A^TA + \lambda^2 I & a_i \\
    a_i^T & 1
  \end{bmatrix}
  \begin{bmatrix} x^{(-i)} \\ \gamma \end{bmatrix} =
  \begin{bmatrix} A^T b - a_i b_i \\ 0 \end{bmatrix}.
\]
Eliminating $x^{(-i)}$ gives
\[
  (1-\ell_i^2) \gamma = \ell_i^2 b_i - a_i^T x^*
\]
where $\ell_i^2 = a_i^T (A^T A + \lambda^2 I)^{-1} a_i$ is called the
{\em leverage score} for row $i$.  Now, observe that
if $r = b-Ax^*$ is the residual for the full problem, then
\[
(1-\ell_i^2) r^{(-i)}
  = (1-\ell_i^2) (b_i + \gamma)
  = (1-\ell_i^2) b_i + \ell_i^2 b_i - a_i^T x_*
  = r_i,
\]
or, equivalently
\[
  r^{(-i)} = \frac{r_i}{1-\ell_i^2}.
\]
We finish the job by observing that $\ell_i^2$ is the $i$th diagonal
element of the orthogonal projector $\Pi = A(A^TA + \lambda I)A^{-1}$, which we
can also write in terms of the economy QR decomposition
\[
  \begin{bmatrix} A \\ \lambda I \end{bmatrix} =
  \begin{bmatrix} Q_1 \\ Q_2 \end{bmatrix} R
\]
as $\Pi = Q_1 Q_1^T$.  Hence, $\ell_i^2$ is the squared row sum of $Q_1$ in
the QR factorization.

\section{Linearly constrained case}

Consider the weighted least squares problem
\[
  \mbox{minimize } \sum_{i=1}^m w_i r_i^2
\]
where $w_1$ is much larger than the others.  If we
let $w_1 \rightarrow \infty$ while the others are fixed, what happens?
We essentially say that we care about enforcing the first equation
above all others, and in the limit we are solving the {\em constrained}
least squares problem
\[
  \mbox{minimize } \sum_{i=2}^m w_i r_i^2 \mbox{ s.t. } r_1 = 0.
\]
Unfortunately, if we actually try to compute this way, we are dancing on
dangerous ground; as $w_1$ goes to infinity, so does the condition
number of the least squares problem.  But this is only an issue with the
weighted formulation; we can formulate the constrained problem in other
ways that are perfectly well-behaved.

In the remainder of this section, we address two ways of handling
the linearly constrained least squares problem
\[
  \mbox{minimize } \|Ax-b\|^2 \mbox{ s.t. } C^T x = d,
\]
by either eliminating variables (the {\em null-space method}) or adding
variables (the method of {\em Lagrange multipliers}).

\subsection{Null space method}

In the null space method, we write an explicit expression for the solutions
to $C^T x = d$ in the form $x^p + W z$ where $x^p$ is a particular solution
to $C^T x^p = d$ and $W$ is a basis for the null space of $C^T$.  Perhaps the
simplest particular solution is $x^p = (C^T)^\dagger d$, the solution with
minimal norm; we can compute both
this particular solution and an orthogonormal null space basis quickly
using a full QR decomposition of $C$:
\[
  C =
    \begin{bmatrix} Q_1 & Q_2 \end{bmatrix}
    \begin{bmatrix} R_1 \\ 0 \end{bmatrix}, \quad
  x^p = Q_1 R_1^{-T} d, \quad W = Q_2.
\]
Note that
\[
  C^T x^p = (R_1^T Q_1^T) x^p = d,
\]
so this is indeed a particular solution.
Having written an explicit parameterization for all solutions of the
constraint equations, we can minimize the least squares objective with
respect to the reduced set of variables
\[
  \mbox{minimize } \|A(x^p + Wz) - b\|^2 = \|(AW)z - (b-Ax^p)\|^2.
\]
This new least squares problem involves a smaller set of variables
(which is good); but in general, even if $A$ is sparse, $AW$ will not be.
So it is appropriate to have a few more methods in our arsenal.

\subsection{Lagrange multipliers}

An alternate method is the method of {\em Lagrange multipliers}.
This is an algebraic technique for adding equations to enforce constraints.

One way to approach the Lagrange multiplier method is to look at the
equations for a constrained minimum.  In order not to have a downhill
direction, we require that the directional derivatives be zero in any
direction consistent with the constraint; that is, we require $Cx = d$
and
\[
  \delta x^T A^T r = 0 \mbox{ when } C^T \delta x = 0.
\]
The constraint says that admissible $\delta x$ are
orthogonal to the columns of $C$; the objective tells us the admissible
$\delta x$ should be orthogonal to the residual.  So we need that $A^T r$
should lie in the column span of $C$; that is,
\[
  A^T r = -C \lambda
\]
for some $\lambda$, and $Cx = d$.  Putting this together,
we have the KKT equations
\[
  \begin{bmatrix}
    A^T A & C \\
    C^T & 0
  \end{bmatrix}
  \begin{bmatrix} x \\ \lambda \end{bmatrix} =
  \begin{bmatrix} A^T b \\ d \end{bmatrix}.
\]

These bordered normal equations are not the end point for constrained
least squares  with Lagrange multipliers, any more than the normal
equations are the end point for unconstrained least squares.  Rather,
we can use this as a starting point for clever manipulations involving
our favorite factorizations (QR and SVD) that reduce the bordered system
to a more computationally convenient form.

\section{Quadratically constrained least squares}

We end the lecture by commenting on the {\em quadratically
  constrained} least squares problem
\[
  \mbox{minimize } \frac{1}{2} \|Ax-b\|^2 \mbox{ s.t.~} \|x\|_M^2 \leq \rho^2
\]
for some positive definite matrix $M$.  Again applying the method of
Lagrange multipliers, we have that either $\|A^\dagger b\|_M^2 \leq \rho^2$ 
(i.e.~the constraint is inactive) or we seek a stationary point of
\[
\mathcal{L}(x, \lambda) =
  \frac{1}{2} \|Ax-b\|^2 + \frac{\lambda}{2} (x^T M x - \rho^2),
\]
and taking variations with respect to $x$ gives us
\[
\nabla_x \mathcal{L} =
  A^T (Ax-b) + \lambda M x = (A^T A + \lambda M) x - A^T b = 0.
\]
That is, if the constrained problem is active, we are actually solving
a Tikhonov-regularized least squares problem, with the Lagrange
multiplier serving the role of the regularization parameter.

\section{Iteratively reweighted least squares}

We conclude with a brief example of how least squares can be used as a
building block for related functions.  As an example, consider
replacing the least squares loss with an alternate loss function:
\[
  \mbox{minimize } \sum_{i} \phi(r_i) \mbox{ s.t. } r = Ax-b
\]
where $\phi$ is a continuous symmetric function such that $\phi(0)=0$.
A common example is the {\em Huber} loss function
\[
  \phi_{\delta}(r) =
  \begin{cases}
    \frac{1}{2} r^2, & |r| \leq \delta \\
    \delta \left( |r|-\frac{1}{2}\delta \right), & \mbox{otherwise}.
  \end{cases}
\]
Optimizing the Huber loss is much less sensitive to outliers than the
least squares loss.  Other loss functions, such as the Tukey biweight,
are even less sensitive to outliers (but are nonconvex, and may lead
to a non-unique optimization problem).

How do we minimize the Huber loss?  There are several options, but one
of the most popular is the {\em iteratively reweighted least squares}
(IRLS) algorithm.  To derive the algorithm, we write the stationary
conditions as
\[
  \delta r^T (\psi(r) \odot r) = 0
\]
where $\psi(r_i) = \phi_{\delta}'(r_i)/r_i$ is a weight.  In terms of
$x$, we have $\delta r = A \delta x$, and so the stationary conditions
are
\[
  \delta x^T A^T W(r) (Ax-b) = 0
\]
where $W(r)$ is a diagonal matrix with entries $W_{ii}(r) = \psi(r_i)$.
That is, the stationary conditions correspond to a set of normal
equations for a weighted least squares problem!  Unfortunately, we
don't know what the weights are; but we can guess them based on
previous iterates.  That is, we repeatedly solve problems of the form
\[
  \mbox{minimize } \|Ax^{(k+1)}-b\|_{W(r^{(k)})}^2
\]
until convergence.

\end{document}
