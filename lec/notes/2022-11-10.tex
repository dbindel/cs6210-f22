\documentclass[12pt, leqno]{article} %% use to set typesize
\usepackage{fancyhdr}
\usepackage[letterpaper=true,colorlinks=true,linkcolor=black]{hyperref}

\usepackage{amsfonts}
\usepackage{amsmath,amssymb,amsthm}
\usepackage{color}
\usepackage{tikz}
\usepackage{pgfplots}
\usepackage{listings}
%\usepackage{courier}
%\usepackage[utf8]{inputenc}
%\usepackage[russian]{babel}

\lstdefinelanguage{Julia}%
  {morekeywords={abstract,break,case,catch,const,continue,do,else,elseif,%
      end,export,false,for,function,immutable,import,importall,if,in,%
      macro,module,otherwise,quote,return,switch,true,try,type,typealias,%
      using,while},%
   sensitive=true,%
   alsoother={$},%
   morecomment=[l]\#,%
   morecomment=[n]{\#=}{=\#},%
   morestring=[s]{"}{"},%
   morestring=[m]{'}{'},%
}[keywords,comments,strings]%

\lstset{
  numbers=left,
  basicstyle=\ttfamily\footnotesize,
  numberstyle=\tiny\color{gray},
  stepnumber=1,
  numbersep=10pt,
}

\newcommand{\iu}{\ensuremath{\mathrm{i}}}
\newcommand{\bbR}{\mathbb{R}}
\newcommand{\bbC}{\mathbb{C}}
\newcommand{\calV}{\mathcal{V}}
\newcommand{\calE}{\mathcal{E}}
\newcommand{\calG}{\mathcal{G}}
\newcommand{\calW}{\mathcal{W}}
\newcommand{\calP}{\mathcal{P}}
\newcommand{\macheps}{\epsilon_{\mathrm{mach}}}
\newcommand{\matlab}{\textsc{Matlab}}
\newcommand{\uQ}{\underline{Q}}
\newcommand{\uR}{\underline{R}}

\newcommand{\ddiag}{\operatorname{diag}}
\newcommand{\fl}{\operatorname{fl}}
\newcommand{\nnz}{\operatorname{nnz}}
\newcommand{\tr}{\operatorname{tr}}
\renewcommand{\vec}{\operatorname{vec}}

\newcommand{\vertiii}[1]{{\left\vert\kern-0.25ex\left\vert\kern-0.25ex\left\vert #1
    \right\vert\kern-0.25ex\right\vert\kern-0.25ex\right\vert}}
\newcommand{\ip}[2]{\langle #1, #2 \rangle}
\newcommand{\ipx}[2]{\left\langle #1, #2 \right\rangle}
\newcommand{\order}[1]{O( #1 )}

\newcommand{\kron}{\otimes}


\newcommand{\hdr}[1]{
  \pagestyle{fancy}
  \lhead{Bindel, Fall 2022}
  \rhead{Matrix Computations}
  \fancyfoot{}
  \begin{center}
    {\large{\bf #1}}
  \end{center}
  \lstset{language=Julia,columns=flexible}  
}


\begin{document}

\hdr{2022-11-10}

\section{Approximation from a subspace}

Our workhorse methods for solving large-scale systems involve two
key ideas: {\em relaxation} to produce a sequence of ever-better
approximations to a problem, and {\em approximation from a subspace}
assumed to contain a good estimate to the solution (e.g.~the subspace
spanned by iterates of some relaxation method).  Having dealt with the
former, we now deal with the latter.

Suppose we wish to estimate the solution to a linear system $Ax^{(*)} = b$ by
an approximate solution $\hat{x} \in \calV$, where $\calV$ is some
approximation subspace.  How should we choose $\hat{x}$?  There are
three standard answers:
\begin{itemize}
\item {\em Least squares}: Minimize $\|A\hat{x}-b\|_M^2$ for some $M$.
\item {\em Optimization}: If $A$ is SPD, minimize $\phi(x) = \frac{1}{2} x^T A x - x^T b$ over $\calV$.
\item {\em Galerkin}: Choose $A\hat{x}-b \perp \calW$ for some test space $\calW$.  In {\em Bubnov-Galerkin}, $\calW = \calV$; otherwise we have
a {\em Petrov-Galerkin} method.
\end{itemize}
These three methods are the standard approaches used in all the methods
we will consider.  Of course, they are not the only possibilities.
For example, we might choose $\hat{x}$ to minimize the residual in
some non-Euclidean norm, or we might more generally choose $\hat{x}$
by optimizing some non-quadratic loss function.  But these approaches lead
to optimization problems that cannot be immediately solved by linear
algebra methods.

The three approaches are closely connected in many ways:
\begin{itemize}
\item
  Suppose $\hat{x}$ is the least squares solution.  Then the normal
  equations give that $A\hat{x}-b \perp M A\calV$; this is a
  (Petrov-)Galerkin condition.
\item
  Similarly, suppose $\hat{x}$ minimizes $\phi(x)$ over the space
  $\calV$.  Then for any $\delta x \in \calV$ we must have
  \[
    \delta \phi = \delta x^T (Ax-b) = 0,
  \]
  i.e.~$Ax-b \perp \calV$.  This is a (Bubnov-)Galerkin condition.
\item
  If $x$ is the least squares solution, then by definition
  we minimize
  \[
    \frac{1}{2} \|Ax-b\|_M^2 = \frac{1}{2} x^T A^T M A x - x^T A^T M b + \frac{1}{2} b^T M b,
  \]
  i.e.~we have the optimization objective for the normal equation
  SPD system $A^T M A x - A^T M b = 0$, plus a constant.
\item
  Note that if $A$ is SPD, then we can express $\phi$ with respect to
  the $A^{-1}$ norm as
  \[
    \phi(x) = \frac{1}{2} \|Ax-b\|_{A^{-1}}^2 - \frac{1}{2} b^T A^{-1} b,
  \]
  so choosing $\hat{x}$ by minimizing $\phi(x)$ is equivalent to
  minimizing the $A^{-1}$ norm of the residual.
\item
  Alternately, write $\phi(x)$ as
  \[
    \phi(x) = \frac{1}{2} \|x-A^{-1} b\|_A^2 - \frac{1}{2} b^T A^{-1} b,
  \]
  and so choosing $\hat{x}$ by minimizing $\phi(x)$ is also
  equivalent to minimizing the $A$ norm of the error.
\end{itemize}
When deriving methods, it is frequently convenient to turn to one or
the other of these characterizations.  But for computation and analysis,
we will generally turn to the Galerkin formalism.

In order for any of these methods to produce accurate results, we need
two properties to hold:
\begin{itemize}
\item {\em Consistency}: Does the space contain a good approximation to $x$?
\item {\em Stability}: Will our scheme find something close to the best
  approximation possible from the space?
\end{itemize}
We leave the consistency and the choice of subspaces to later; for now,
we deal with the problem of method stability.

\section{Quasi-optimality}

We quantify the stability of a subspace approximation method via
a {\em quasi-optimality bound}:
\[
  \|x^*-\hat{x}\| \leq C \min_{v \in \calV} \|x^*-v\|.
\]
That is, the approximation $\hat{x}$ is quasi-optimal if it has
error within some factor $C$ of the best error possible within the
space.

To derive quasi-optimality results, it is useful to think of all
of our methods as defining a {\em solution projector}
that maps $x^*$ to the approximate solution to $A\hat{x} = Ax^* = b$.
From the (Petrov-)Galerkin perspective, if $W \in \bbR^{n \times k}$
and $V \in \bbR^{n \times k}$ are bases for the trial space $\calW$
and $\calV$, respectively, then we have
\begin{align*}
  W^T A V \hat{y} &= W^T b, \quad \hat{x} = V \hat{y} \\
  \hat{x} &= V (W^T A V)^{-1} W^T b \\
          &= V (W^T A V)^{-1} W^T A x^*. \\
          &= \Pi x^*.
\end{align*}
The {\em error projector} $I-\Pi$ maps $x^*$ to the error $\hat{x}-x^*$
in approximately solving $A\hat{x} \approx Ax^* = b$.  There is no
error iff $x^*$ is actually in $\calV$; that is, $\calV$ is the
null space of $I-\Pi$.  Hence, if $\tilde{x}$ is any vector in $\calV$,
then
\[
  \hat{e} = (I-\Pi) x = (I-\Pi) (x-\tilde{x}) = (I-\Pi) \tilde{e}.
\]
Therefore we have
\[
  \|x-\hat{x}\| \leq \|I-\Pi\| \min_{\tilde{x} \in \calV} \|x-\tilde{x}\|,
\]
and a bound on $\|I-\Pi\|$ gives a quasi-optimality result.

For any operator norm, we have
\[
  |I-\Pi\| \leq 1+\|\Pi\| \leq 1 + \|V\| \|(W^T A V)^{-1}\| \|W^T A\|;
\]
and in any Euclidean norm, if $V$ and $W$ are chosen to have orthonormal
columns, then
\[
  \|I-\Pi\| \leq 1 + \|(W^T A V)^{-1}\| \|A\|.
\]
If $A$ is symmetric and positive definite and $V = W$, then the
interlace theorem gives $\|(V^T A V)^{-1}\| \leq \|A^{-1}\|$,
and the quasi-optimality constant is bounded by $1 + \kappa(A)$.
In more general settings, though, we may have no guarantee that
the projected matrix $W^T A V$ is far from singular, even if $A$
itself is nonsingular.  To guarantee boundedness of $(W^T A V)^{-1}$
{\em a priori} requires a compatibility condition relating
$\calW$, $\calV$, and $A$; such a condition is sometimes called
the {\em LBB} condition
(for Ladyzhenskaya-Babu\v{s}ka-Brezzi) or
the {\em inf-sup} condition, so named because (as we have discussed
previously)
\[
  \sigma_{\min}(W^T A V) =
  \inf_{w \in \calW} \sup_{v \in \calV} \frac{w^T A v}{\|w\| \|v\|}.
\]
The LBB condition plays an important role when Galerkin methods are
used to solve large-scale PDE problems, since there it is easy to
choose the spaces $\calV$ and $\calW$ in a way that leads to very
bad conditioning.  But for iterative solvers of the type we discuss
in this course (Krylov subspace solvers), such pathologies are a more
rare occurrence.  In this setting, we may prefer to
monitor $\|(W^T A V)^{-1}\|$ directly as we go along, and to simply
increase the dimension of the space if we ever run into trouble.

\section{Krylov subspaces}

The {\em Krylov subspace} of dimension $k$ generated by
$A \in \bbR^{n \times n}$ and $b \in \bbR^n$ is
\[
  \mathcal{K}_k(A,b)
    = \operatorname{span}\{ b, Ab, \ldots, A^{k-1} b \}
    = \{ p(A) b : p \in \mathcal{P}_{k-1} \}.
\]
Krylov subspaces are a natural choice for subspace-based methods for
approximate linear solves, for two reasons:
\begin{itemize}
\item If all you are allowed to do with $A$ is compute matrix-vector
  products, and the only vector at hand is $b$, what else would you do?
\item The Krylov subspaces have excellent approximation properties.
\end{itemize}

Krylov subspaces have several properties that are worthy of comment.
Because the vectors $A^{j} b$ are proportional to the vectors obtained
in power iteration, one might reasonably (and correctly)
assume that the space quickly contains good approximations to the
eigenvectors associated with the largest magnitude eigenvalues.
Krylov subspaces are also {\em shift-invariant}, i.e. for any $\sigma$
\[
  \mathcal{K}_k(A-\sigma I, b) = \mathcal{K}_k(A,b).
\]
By choosing different shifts, we can see that the Krylov subspaces
tend to quickly contain not only good approximations to the eigenvector
associated with the largest magnitude eigenvalue, but to all
``extremal'' eigenvalues.

Most arguments about the approximation properties of Krylov subspaces
derive from the characterization of the space as all vectors $p(A) b$
where $p \in \mathcal{P}_{k-1}$ and from the spectral mapping theorem,
which says that if $A = V \Lambda V^{-1}$ then
$p(A) = V p(\Lambda) V^{-1}$.  Hence, the distance between
an arbitrary vector (say $d$) and the Krylov subspace is
\[
  \min_{p \in \mathcal{P}_{k-1}}
  \left\| V \left[ p(\Lambda) V^{-1} b - V^{-1} d \right] \right\|.
\]
As a specific example, suppose that we want to choose $\hat{x}$
in a Krylov subspace in order to minimize the residual $A \hat{x} - b$.
Writing $\hat{x} = p(A) b$, we have that we want to minimize
\[
  \|[A p(A)-I] b\| = \|q(A) b\|
\]
where $q(z)$ is a polynomial of degree at most $k$ such that $q(1) = 1$.
The best possible residual in this case is bounded by
\[
  \|q(A) b\| \leq \kappa(V) \|q(\Lambda)\| \|b\|,
\]
and so the relative residual can be bounded in terms of the condition
number of $V$ and the minimum value that can bound $q$ on the spectrum
of $A$ subject to the constraint that $q(0) = 1$.

\section{Chebyshev polynomials}

Suppose now that $A$ is symmetric positive definite, and we seek to
minimize $\|q(A) b\| \leq \|q(\Lambda)\| \|b\|$.  Controlling $q(z)$
on all the eigenvalues is a pain, but it turns out to be simple to
instead bound $q(z)$ over some interval $[\alpha_1, \alpha_n]$
The polynomial we want is the {\em scaled and shifted Chebyshev polynomial}
\[
  q_m(z) =
  \frac{T_m\left( (z-\bar{\alpha})/\rho \right)}
       {T_m\left( -\bar{\alpha}/\rho \right)}
\]
where $\bar{\alpha} = (\alpha_n + \alpha_1)/2$ and
$\rho = (\alpha_n-\alpha_1)/2$.

The Chebyshev polynomials $T_m$ are defined by the recurrence
\begin{align*}
  T_0(x) &= 1 \\
  T_1(x) &= x \\
  T_{m+1}(x) &= 2x T_m(x) - T_{m-1}(x), \quad m \geq 1.
\end{align*}
The Chebyshev polynomials have a number of remarkable properties, but
perhaps the most relevant in this setting is that
\[
  T_m(x) =
  \begin{cases}
    \cos(m \cos^{-1}(x)), & |x| \leq 1, \\
    \cosh(m \cosh^{-1}(x)), &|x| \geq 1
  \end{cases}.
\]
Thus, $T_m(x)$ oscillates between $\pm 1$ on the interval $[-1,1]$,
and then grows very quickly outside that interval.  In particular,
\[
  T_{m}(1 + \epsilon) \geq \frac{1}{2} (1+m\sqrt{2\epsilon}).
\]
Thus, we have that on $[\alpha_, \alpha_n]$,
$|q_m| \leq \frac{2}{1+m\sqrt{2\epsilon}}$
where
\[
  \epsilon = \bar{\alpha}/\rho-1
  = \frac{2\alpha_1}{\alpha_n-\alpha_1}
  = 2 \left( \kappa(A)-1 \right)^{-1},
\]
and hence
\begin{align*}
  |q_m(z)|
  &\leq \frac{2}{1+2m/\sqrt{\kappa(A)-1}} \\
  &= 2\left( 1- \frac{2m}{\sqrt{\kappa(A)-1}}\right) + O\left(\frac{m^2}{\kappa(A-1)}\right).
\end{align*}
Hence, we expect to reduce the optimal residual in this case
by at least about $2/\sqrt{\kappa(A)-1}$ at each step.

\section{Chebyshev: Uses and Limitations}

We previously sketched out an approach for analyzing the convergence of
methods based on Krylov subspaces:
\begin{enumerate}
\item
  Characterize the Krylov subspace of interest in terms of polynomials,
  i.e. $\mathcal{K}_k(A,b) = \{ p(A)b : p \in \mathcal{P}_{k-1} \}$.
\item
  For $\hat{x} = p(A) b$, write an associated error (or residual)
  in terms of a related polynomial in $A$.
\item
  Phrase the problem of minimizing the error, residual, etc.~in terms
  of minimizing a polynomial $q(z)$ on the spectrum of $A$
  (call this $\Lambda(A)$).  The polynomial $q$ must generally satisfy
  some side constraints that prevent the zero polynomial from being
  a valid solution.
\item
  Let $\Lambda(A) \subset \Omega$, and write
  \[
    \max_{\lambda \in \Lambda(A)} |q(\lambda)| \leq
    \max_{z \in \Omega} |q(z)|.
  \]
  The set $\Omega$ should be simpler to work with than the set of
  eigenvalues.  The simplest case is when $A$ is symmetric positive
  definite and $\Omega = [\lambda_1, \lambda_n]$.
\item
  The optimization problem can usually be phrased in terms of special
  polynomial families.  The simplest case, when $\Omega$ is just an
  interval, usually leads to an analysis via Chebyshev polynomials.
\end{enumerate}
The analysis sketched above is the basis for the convergence analysis
of the Chebyshev semi-iteration, the conjugate gradient method, and
(with various twists) several other Krylov subspace methods.

The advantage of this type of analysis is that it leads to convergence
bounds in terms of some relatively simple property of the matrix, such
as the condition number.  The disadvantage is that the approximation of
the spectral set $\Lambda(A)$ by a bounding region $\Omega$ can lead to
rather pessimistic bounds.  In practice, the extent to which we are able
to find good solutions in a Krylov subspace often depends on the
``clumpiness'' of the eigenvalues.  Unfortunately, this ``clumpiness''
is rather difficult to reason about a priori!  Thus, the right way to evaluate
the convergence of Krylov methods in practice is usually to try them out,
plot the convergence curves, and see what happens.

\end{document}
