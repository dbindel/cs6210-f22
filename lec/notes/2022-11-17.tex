\documentclass[12pt, leqno]{article} %% use to set typesize
\usepackage{fancyhdr}
\usepackage[letterpaper=true,colorlinks=true,linkcolor=black]{hyperref}

\usepackage{amsfonts}
\usepackage{amsmath,amssymb,amsthm}
\usepackage{color}
\usepackage{tikz}
\usepackage{pgfplots}
\usepackage{listings}
%\usepackage{courier}
%\usepackage[utf8]{inputenc}
%\usepackage[russian]{babel}

\lstdefinelanguage{Julia}%
  {morekeywords={abstract,break,case,catch,const,continue,do,else,elseif,%
      end,export,false,for,function,immutable,import,importall,if,in,%
      macro,module,otherwise,quote,return,switch,true,try,type,typealias,%
      using,while},%
   sensitive=true,%
   alsoother={$},%
   morecomment=[l]\#,%
   morecomment=[n]{\#=}{=\#},%
   morestring=[s]{"}{"},%
   morestring=[m]{'}{'},%
}[keywords,comments,strings]%

\lstset{
  numbers=left,
  basicstyle=\ttfamily\footnotesize,
  numberstyle=\tiny\color{gray},
  stepnumber=1,
  numbersep=10pt,
}

\newcommand{\iu}{\ensuremath{\mathrm{i}}}
\newcommand{\bbR}{\mathbb{R}}
\newcommand{\bbC}{\mathbb{C}}
\newcommand{\calV}{\mathcal{V}}
\newcommand{\calE}{\mathcal{E}}
\newcommand{\calG}{\mathcal{G}}
\newcommand{\calW}{\mathcal{W}}
\newcommand{\calP}{\mathcal{P}}
\newcommand{\macheps}{\epsilon_{\mathrm{mach}}}
\newcommand{\matlab}{\textsc{Matlab}}
\newcommand{\uQ}{\underline{Q}}
\newcommand{\uR}{\underline{R}}

\newcommand{\ddiag}{\operatorname{diag}}
\newcommand{\fl}{\operatorname{fl}}
\newcommand{\nnz}{\operatorname{nnz}}
\newcommand{\tr}{\operatorname{tr}}
\renewcommand{\vec}{\operatorname{vec}}

\newcommand{\vertiii}[1]{{\left\vert\kern-0.25ex\left\vert\kern-0.25ex\left\vert #1
    \right\vert\kern-0.25ex\right\vert\kern-0.25ex\right\vert}}
\newcommand{\ip}[2]{\langle #1, #2 \rangle}
\newcommand{\ipx}[2]{\left\langle #1, #2 \right\rangle}
\newcommand{\order}[1]{O( #1 )}

\newcommand{\kron}{\otimes}


\newcommand{\hdr}[1]{
  \pagestyle{fancy}
  \lhead{Bindel, Fall 2022}
  \rhead{Matrix Computations}
  \fancyfoot{}
  \begin{center}
    {\large{\bf #1}}
  \end{center}
  \lstset{language=Julia,columns=flexible}  
}

\newcommand{\calK}{\mathcal{K}}

\begin{document}

\hdr{2022-11-17}

\section{References}

There is a lot of ground to cover when it comes to Krylov subspace
methods, and we scarcely have time to do justice to the two most popular
Krylov subspace methods (CG for the SPD case and GMRES elsewhere).
Apart from the material in Golub and Van Loan and other standard texts,
I highly recommend two books for a survey of other methods and some
practical details:
\begin{enumerate}
\item \href{http://www.netlib.org/templates/templates.pdf}{\em Templates for the Solution of Linear Systems:
Building Blocks for Iterative Methods}
  by Barrett {\em et al}.  This is freely available, and includes
  what you need to know to get started with various methods.
\item {\em Iterative methods for sparse linear systems} by Y.~Saad.
  This is now in a second edition (available from SIAM), but you
  can also get the first edition at \href{http://www-users.cs.umn.edu/~saad/books.html}{Saad's web page}.
\item {\em Iterative methods for solving linear systems} by Anne Greenbaum
  (published by SIAM) is a fairly up-to-date treatment of the major iterative
  solvers for linear systems, including the whole family of Krylov subspace
  solvers as well as classical stationary iterations and multigrid methods.
\item {\em Iterative methods for linear and nonlinear equations} by C.~T.~Kelley
  is another SIAM book --- are you seeing a theme?
  It covers CG and GMRES, though not the other
  Krylov iterations; however, it also covers nonlinear iterations.  It is
  short and tutorial in nature.
\end{enumerate}
For the conjugate gradients method, another popular introduction is
Shewchuk's ``\href{http://www.cs.cmu.edu/~quake-papers/painless-conjugate-gradient.pdf}{Introduction to the Conjugate Gradient Method Without the Agonizing Pain}''.


\section{Conjugate gradients}

We now turn to the method of conjugate gradients (CG), perhaps the
best known of the Krylov subspace solvers.  The CG iteration can be
characterized as the iteration for a symmetric positive definite
$A$ that minimizes the energy
\[
  \phi(x) = \frac{1}{2} x^T A x - x^T b
\]
over a Krylov subspace; as we have already seen,
\[
  \phi(x) + \frac{1}{2} b^T A^{-1} b
  = \frac{1}{2} \|x-A^{-1} b\|_A^2
  = \frac{1}{2} \|Ax-b\|_{A^{-1}}^2,
\]
so this minimization corresponds to minimizing the error in the $A$-norm
or the residual in the $A^{-1}$ norm.  We also have seen the shape of
the standard error analysis, which involves looking at a Chebyshev
polynomial on an interval containing the spectrum.  The iteration turns
out to be forward unstable, so the behavior in floating point arithmetic
is not the same as the behavior in theory; but this does not prevent
the iteration from being highly effective, partly because we can
write the iteration in a form that involves an explicit residual,
and looking at a freshly-computed residual gives the method a
self-correcting property.

Our goal for today is to look at the mechanics of the method.

\subsection{CG via Lanczos}

Last time, we discussed the Lanczos iteration, which produces the
Lanczos decomposition
\[
  A Q_k = Q_{k+1} \bar{T}_k
\]
via the iteration
\[
  \beta_k q_{k+1} = A q_k - \alpha_k q_k - \beta_{k-1} q_{k-1}
\]
where $\alpha_k = q_k^T A q_k$.  One of the simplest derivations
for the {\em conjugate gradient} (CG) method is in terms of the
Lanczos decomposition.

In terms of the energy
\[
  \phi(x) = \frac{1}{2} x^T A x - x^T b,
\]
the problem of finding the ``best'' (minimum energy) approximate
solution in the space becomes
\[
  \mbox{minimize } \phi(Q_k y_k) = \frac{1}{2} y_k^T T_k y_k - y_k^T e_1 \|b\|,
\]
which is solved by
\[
  T_k y_k = e_1 \|b\|.
\]

Now let us suppress the indices for a moment and
write $T = LU$ (which can be computed stably without pivoting,
as $T$ is SPD).  Then we can write the approximate solution
$\hat{x}$ as
\[
  \hat{x} = Q U^{-1} L^{-1} e_1 \|b\|,
\]
which we will group as
\[
  \hat{x} = V \hat{y}, \quad VU = Q, \quad Ly = e_1 \|b\|.
\]
Solving the system for $y$ by forward substitution yields
\begin{align*}
  y_1 &= \|b\| \\
  y_k &= -l_{k,k-1} y_{k-1}.
\end{align*}
Similarly, we can compute the columns of $V$ by forward
substitution:
\begin{align*}
  v_1 &= q_1 / u_{11} \\
  v_k &= \frac{1}{u_{kk}} \left( q_k - v_{k-1} u_{k-1,k} \right).
\end{align*}
The advantage of this formulation is that if we extend the Krylov
subspace, we simply extend the tridiagonal (and associated
factorization), add another component to $y$, and bring in a new
vector $v$ --- all without disturbing the computations done before.
Hence, we have a sequence of coupled recurrences for the columns
of $Q$ and of $V$ that allow us to incrementally update the solution
at the cost of a matrix-vector multiply and a constant amount
of vector arithmetic per step.

This is a useful approach, but it does not shed much insight into how the
method could be extended to optimize more general objectives than
quadratics.  For that, we need the approach that gives the CG method its
name.

\subsection{Another approach to CG}

An alternate approach to the conjugate gradient method does not directly
invoke Lanczos, but instead relies on properties that must be satisfied
at each step by the residual $r_m = b-Ax_m$ and the update
$d_m = x_{m+1}-x_m$.  We assume throughout that $x_m$ is drawn from
$\calK_m(A,b)$, which implies that $r_m \in \calK_{m+1}(A,b)$ and
$d_m \in \calK_{m+1}(A,b)$.

First, note that $r_m \perp \calK_{m}(A,b)$ and $d_m \perp_A
\calK_{m}(A,b)$.\footnote{
  $u \perp_A v$ means $u$ and $v$ are orthogonal
  in the $A$-induced inner product, i.e. $u^T A v = 0$.
}  The former statement comes from the Galerkin
criterion in the previous section.  The latter statement comes from
recognizing that $r_{m+1} = Ad_m + r_m \perp \calK_{m}(A,b)$; with
Galerkin condition $r_m \perp \calK_{m}(A,b)$, this means $Ad_m \perp
\calK_{m}(A,b)$.  Together, these statements give us $r_m$ and $d_m$
to within a scalar factor, since there is only one direction in
$\calK_{m+1}(A,b)$ that is orthogonal to all of $\calK_{m}(A,b)$, and
similarly there is only one direction that is $A$-orthogonal.
This suggests the following idea to generate the sequence of
approximate solutions $x_k$:
\begin{enumerate}
\item
  Find a direction $p_{k-1} \in \calK_{k}(A,b)$ that is $A$-orthogonal
  to $\calK_{k-1}(A,b)$.
\item
  Compute $x_{k} = x_{k-1} + \alpha_k p_{k-1}$ so that
  \[
    r_k = r_{k-1} - \alpha_k A p_{k-1} \perp r_{k-1},
  \]
  i.e. set $\alpha_k = (r_{k-1}^T r_{k-1}) / (p_{k-1}^T A p_{k-1})$.
  Orthogonality to the rest of $\calK_{k}(A,b)$ follows
  automatically from the construction.
\item
  Take $r_k \in \calK_{k+1}(A,b)$ and $A$-orthogonalize against
  everything in $\calK_{k}(A,b)$ to generate the new direction $p_k$.
  As with the Lanczos procedure, the real magic in this idea is that
  we have to do very little work to generate $p_k$ from $r_k$.
  Note that for any $j < k-1$, we have $p_j^T A r_k = (Ap_j)^T r_k = 0$,
  because $Ap_j \in \calK_{j+2}(A,b) \subset \calK_k(A,b)$ is automatically
  orthogonal to $r_k$.  Therefore, we really only need to choose
  \[
    p_k = r_k + \beta p_{k-1},
  \]
  such that $p_{k-1}^T A p_k$, i.e.
  $\beta_k = -(p_{k-1}^T A r_k)/(p_{k-1}^T A p_{k-1})$.
  Note, though, that $Ap_{k-1} = -(r_{k}-r_{k-1})/\alpha_k$;
  with a little algebra, we find
  \[
    \beta_k = -\frac{r_k^T Ap_k}{p_{k-1}^T A p_{k-1}}
            = \frac{(r_k^T r_k)/\alpha_k}{r_{k-1}^T r_{k-1}/\alpha_k}
            = \frac{r_k^T r_k}{r_{k-1}^T r_{k-1}}.
  \]
\end{enumerate}

Putting everything together, we have the following coupled recurrences
for the solutions $x_k$, residuals $r_k$, and search directions $p_k$:
\begin{align*}
  \alpha_k &= (r_{k-1}^T r_{k-1})/(p_{k-1}^T A p_{k-1}) \\
  x_{k} &= x_{k-1} + \alpha_k p_{k-1} \\
  r_{k} &= r_{k-1} - \alpha_k A p_{k-1} \\
  \beta_{k} &= (r_k^T r_k)/(r_{k-1}^T r_{k-1}) \\
  p_k &= r_{k} + \beta_k p_{k-1}.
\end{align*}
The sequences $r_k$ and $p_k$ respectively form orthogonal and $A$-orthogonal
bases for the nested Krylov subspaces generated by $A$ and $b$.

\subsection{Preconditioning}

What happens if we want to compute not on the space $\mathcal{K}_k(A,b)$,
but the preconditioned space $\mathcal{K}_k(M^{-1} A, M^{-1} b)$
where $M$ is some symmetric positive definite matrix?  Unfortunately,
we cannot apply CG directly to a system involving $M^{-1} A$, since
even if $M$ and $A$ are SPD, the product will generally not be.
On the other hand, we can certainly work with the related system
\[
  (M^{-1/2} A M^{-1/2}) (M^{1/2} x) = M^{-1/2} b.
\]
This is a symmetric positive definite system, and the eigenvalues
of $M^{-1/2} A M^{-1/2}$ are the same as the generalized eigenvalues
of the pencil $(A,M)$.  Moreover, we can work with this system
{\em implicitly} without ever having to form the awkward square
root.

Define $\bar{p}_k = M^{-1/2} p_k$ and $\bar{r}_k = M^{1/2} r_k$;
then CG iteration on the related system can be rephrased as
\begin{align*}
  \alpha_k &= (\bar{r}_{k-1}^T M^{-1} \bar{r}_{k-1})/(\bar{p}_{k-1}^T A \bar{p}_{k-1}) \\
  x_{k} &= x_{k-1} + \alpha_k \bar{p}_{k-1} \\
  \bar{r}_{k} &= \bar{r}_{k-1} - \alpha_k A \bar{p}_{k-1} \\
  \beta_{k} &= (\bar{r}_k^T M^{-1} \bar{r}_k)/(\bar{r}_{k-1}^T M^{-1} \bar{r}_{k-1}) \\
  \bar{p}_k &= M^{-1} \bar{r}_{k} + \beta_k \bar{p}_{k-1}.
\end{align*}
Because expressions involving $M^{-1}$ and the residual appear throughout,
we introduce a new variable $z_k = M^{-1} r_k$, leading to
\begin{align*}
  \alpha_k &= (\bar{r}_{k-1}^T z_{k-1})/(\bar{p}_{k-1}^T A \bar{p}_{k-1}) \\
  x_{k} &= x_{k-1} + \alpha_k \bar{p}_{k-1} \\
  \bar{r}_{k} &= \bar{r}_{k-1} - \alpha_k A \bar{p}_{k-1} \\
  M z_{k} &= r_k \\
  \beta_{k} &= (\bar{r}_k^T z_k)/(\bar{r}_{k-1}^T z_{k-1}) \\
  \bar{p}_k &= z_{k} + \beta_k \bar{p}_{k-1}.
\end{align*}

Another way of thinking about the preconditioned CG iteration is
that it is ordinary CG, whether thought of in terms of conjugate
directions or in terms of Lanczos, but with a different inner
product: the $M^{-1}$ inner product on residuals, or the $M$ inner
product in the Lanczos procedure.

\subsection{Nonlinear CG}

One of the advantages of the interpretation of CG in terms of search
directions and residuals is that it generalizes beyond the case of
quadratic optimization or linear system solving to more general
optimization problems.  To derive nonlinear CG, we generalize the
quantities in the ordinary CG iteration in the following way:
\begin{itemize}
\item
  In ordinary CG, we let $\phi$ be a quadratic energy function.
  In nonlinear CG, $\phi$ is a more general (though ideally
  convex) objective function.
\item
  In ordinary CG, we have $r_k = -\nabla \phi(x_k) = b-Ax_k$.
  In nonlinear CG, we take $r_k = -\nabla \phi(x_k)$, though the
  gradient expression will generally be more complicated.
\item
  In ordinary CG, we choose a search direction $p_k = r_k + \beta_k p_{k-1}$
  where $\beta_k = r_k^T r_k / r_{k-1}^T r_{k-1}$.  In nonlinear CG,
  we may use the same formula (the {\em Fletcher-Reeves} formula),
  or we may choose any number of other formulas that are equivalent
  in the quadratic case but not in the more general case.
\item
  In ordinary CG, once we choose a search direction $p_{k-1}$, we compute
  a step $x_k = x_{k-1} + \alpha_{k} p_{k-1}$.  The $\alpha_k$ has
  the property
  \[
    \alpha_k = \operatorname{argmin}_{\alpha} \phi(x_k + \alpha p_{k-1})
  \]
  In nonlinear CG, we instead use a line search to choose the step size.
\end{itemize}
Like ordinary CG, nonlinear CG iterations can be preconditioned.

\subsection{The many approaches to CG}

The description I have given in these notes highlights (I hope)
how orthogonality of the residuals and $A$-orthogonality
of search directions follows naturally from the Galerkin condition,
and how the rest of the CG iteration can be teased out of these
orthogonality relations.  However, this is far from the only way
to ``derive'' the method of conjugate gradients.  The discussion
given by Demmel and by Saad
(in {\em Iterative Methods for Sparse Linear Systems})
highlights the Lanczos connection, and uses this connection to
show the existence of $A$-orthogonal search directions.
Golub and Van Loan show the Lanczos connection, but also show
how conjugate gradients can be derived as a general-purpose minimization
scheme applied to the quadratic function $\phi(x)$.  Trefethen and Bau
give the iteration without derivation first, and then gradually explain
some of its properties.  If you find these discussions confusing,
or simply wish to read something amusing, I recommend Shewchuk's
``\href{http://www.cs.cmu.edu/~quake-papers/painless-conjugate-gradient.pdf}{Introduction to the Conjugate Gradient Method Without the Agonizing Pain}''.

\section{GMRES}

The {\em generalized minimal residual} (GMRES) method of solving
linear systems works with general systems of linear equations.
Next to CG, it is probably the second-most popular of the Krylov
subspace iterations.

The GMRES method is so named because it chooses the solution from a
linear subspace that minimizes the (Euclidean) norm of the residual
over successive Krylov subspaces.  In terms of the Arnoldi
decompositions
\[
  AQ_k = Q_{k+1} \bar{H}_k,
\]
we have that $x_k = Q_k y_k$ where
\[
  y_k = \operatorname{argmin}_y \|\bar{H}_k y - \|b\| e_1\|^2.
\]
One can solve the Hessenberg least squares problem in $O(k^2)$
time, but this is generally a non-issue.  The true cost of GMRES
is in saving the basis (which can use memory very quickly) and in
keeping the basis orthogonal.

Unlike the CG method, alas, the GMRES method does not boil down to a
short recurrence through a sequence of clever tricks.  Consequently,
we generally cannot afford to run the iteration for many steps before
{\em restart}.  We usually denote the iteration with periodic
restarting every $m$ steps as GMRES$(m)$.  That is, at each step we
\begin{enumerate}
  \item Start with an initial guess $\hat{x}$ from previous steps.
  \item Form the residual $r = b-A\hat{x}$.
  \item Run $m$ steps of GMRES to approximately solve $Az = r$.
  \item Update $\hat{x} := \hat{x} + z$.
\end{enumerate}

The GMRES iteration is generally used with a preconditioner.
The common default is to use preconditioning on the left, i.e.
solve
\[
  M^{-1} A x = M^{-1} b;
\]
in this setting, GMRES minimizes not the original residual,
but the {\em preconditioned} residual.  To the extent that the
preconditioner reduces the condition number of the problem overall,
the norm of the preconditioned residual tends to be a better indicator
for forward error than the norm of the un-preconditioned residual.
Of course, one can also perform preconditioning on the right (i.e. changing
the unknown), or perform two-sided preconditioning.

The standard GMRES iteration (along with CG and almost every other
Krylov subspace iteration) assumes a single, fixed preconditioner.
But what if we want to try several preconditioners at once, or
perhaps to use Gauss-Southwell or a chaotic relaxation method for
preconditioning?  Or perhaps we want to use a variable number of
steps of some other iteration to precondition something like GMRES?
For this purpose, it is useful to consider the
{\em flexible GMRES} variant (FGMRES).  Though it no longer technically
is restricted to a Krylov subspace generated by a fixed matrix, the
FGMRES iteration looks very similar to the standard GMRES iteration;
we build an Arnoldi-like decomposition with the form
\[
  AZ_m = V_{m+1} \bar{H}_m
\]
and then compute updates as a linear combination of the columns of $Z_m$
by solving a least squares problem with $\bar{H}_m$.  But here, each
column of $Z_m$ looks like $z_j = M_j^{-1} v_j$ where each $M_j$ may
be different.

\section{Bi-Lanczos}

So far, our focus has been on Krylov subspace methods that we can
explain via the Lanczos or Arnoldi decompositions.  The Lanczos-based
CG has many attractive properties, but it only works with symmetric
and positive definite matrices.  One can apply CG to a system of normal
equations --- the so-called CGNE method --- but this comes at the cost
of squaring the condition number.  There are also methods such as the
LSQR iteration that implicitly work with the normal equations, but use
an incrementally-computed version of the Golub-Kahan bi-diagonalization.
The Arnoldi-based GMRES iteration works for more general classes
of problems, and indeed it is the method of choice; but it comes at
a stiff penalty in terms of orthogonalization costs.

Are there alternative methods that use short recurrences (like CG) but
are appropriate for nonsymmetric matrices?  There are several, though
all have some drawbacks; the QMR and BiCG iterations may be the most
popular.  The key to the behavior of these methods comes from their
use of a different decomposition, the {\em bi-orthogonal Lanczos
factorization}:
\begin{align*}
  A Q_j &= Q_j T_j + \beta_{j+1} q_{j+1} e_j^* \\
  A^* P_j &= P_j T_j^* + \bar{\gamma}_{j+1} p_{j+1} e_j^* \\
  P_j^* Q_j &= I.
\end{align*}
Here, the bases $Q_j$ and $P_j$ span Krylov subspaces generated
by $A$ and $A^*$, respectively (which means these algorithms require
not only a function to apply $A$ to a vector, but also a function
to apply $A^*$).  The bases are not orthonormal, and indeed may become
rather ill-conditioned.  They {\em do} have a mutual orthogonality
relationship, though, namely $P_j^* Q_j = I$.

Details of the bi-orthogonal Lanczos factorization and related
iterative algorithms can be found in the references.  For the present,
we satisfy ourseves with a few observations:
\begin{itemize}
\item
  The GMRES iteration shows monotonic reduction in the preconditioned
  residual, even with restarting.  CG shows monotonic reduction in the
  error or residual when measured in appropriate norms.  The methods
  based on bi-orthogonal Lanczos, however, can show rather erratic
  convergence; errors decay in general, but they may exhibit
  intermediate local increases.  BiCG is generally more erratic than
  QMR.
\item
  Even in exact arithmetic, the subspace bases formed by bi-Lanczos
  may become rather ill-conditioned.
\item
  The bi-orthogonal iterations sometimes show {\em breakdown} behavior
  where the local approximation problem becomes singular.  This can be
  overcome using {\em lookahead} techniques, though it complicates the
  algorithm.
\end{itemize}
The relative simplicity of GMRES --- both in theory and in implementation ---
perhaps explains its relative popularity.  Nonetheless, these other methods
are worth knowing about.

\section{Extrapolation and mixing}

When we discussed CG, we also briefly discussed {\em nonlinear} CG
methods (e.g.~Fletcher-Reeves).  One can similarly extend Krylov
subspace ideas to accelerate nonlinear equation solving methods;
that is, given a fixed point iteration
\[
  x^{(k+1)} = G(x^{(k)}),
\]
we can accelerate the computation of the fixed point by taking an
appropriate linear combination of the iterates $x^{(k)}$.  This is
a powerful idea; indeed, it is so powerful that it can be used to
compute repulsive fixed points where the usual iteration would
diverge!  The techniques usually go under the heading of
{\em extrapolation methods} (including Reduced Rank Extrapolation or RRE,
Minimal Polynomial Extrapolation or MPE, and Vector Pad\'e Extrapolation);
and {\em acceleration} or {\em mixing} techniques, the most popular of
which is the {\em Anderson acceleration} method.  Applied to the iterates
of a stationary linear system solver, these techniques are all formally
equivalent to Krylov subspace solvers.  In particular, RRE is equivalent
to GMRES (in exact arithmetic).

The idea behind extrapolation methods is to exploit systematic patterns
in the convergence of fixed point iteration.  For example, suppose the
error iteration gave us (approximately)
\[
  e^{(k)} = \sum_{j=1}^m v^{(j)} \alpha_j^k
\]
where the vectors $v^{(j)}$ and the exponents $\alpha_j$ were unknown.
The hope is that we can learn the parameters of the error iteration
by fitting a model to the update sequence:
\[
  u^{(k)} = x^{(k+1)}-x^{(k)} = e^{(k+1)}-e^{(k)} =
  \sum_{j=1}^m (\alpha_j-1) v^{(j)} \alpha_j^k.
\]
If $p(z) = c_0 + c_1 z + \ldots + c_m z^m$ is a polynomial such
that $p(\alpha_j) = 0$ for each $\alpha_j$, then we should satisfy
\[
  \sum_{j=1}^m c_j u^{(k+j)} = 0.
\]
If we look at enough update steps, we can determine both the
coefficient vectors and the exponents.

With an appropriate model, extrapolation methods can produce rather
astonishing results.  Of course, extrapolation methods are subject
to issues of overfitting, and (particularly when the convergence is
irregular) may produce results that are wildly incorrect.

\subsection{Communication-Avoiding (CA) Krylov}

In highly parallel computing systems, the cost of computing with Krylov
subspaces may be dominated not by the matrix-vector products, but by the
cost of computing dot products for the purpose of orthogonalization.
Repeatedly applying matrix-vector products may involves rather local
communication patterns, but dot products involve a global communication.
Of course, we could (in principle) form a power basis for the Krylov
subspace; but this basis is typically too ill-conditioned for serious
work.  So what is one to do?

The {\em communication-avoiding} Krylov methods use the power of
polynomials to thread between the Scylla of synchronization costs and
the Charybdis of catastrophic ill-conditioning.  In general, we write
Krylov subspace bases as
\[
  \mathcal{K}_k(A,b) = \operatorname{span}\{ p_j(A) b \}_{j=0}^{(k-1)}.
\]
where $p_j(z)$ is a degree $j$ polynomial.  In the case of the
power basis, $p_j(z) = z^j$; and in the case of the Lanczos or
Arnoldi bases, $p_j(z)$ is chosen fully adaptively.  The communication
avoiding approach is to choose $p_j(z)$ in advance, but using information
about the spectra to ensure that the vectors $p_j(A) b$ are not too
nearly co-linear.

As with some of the other topics in this section, the big idea behind
communication-avoiding Krylov methods is simple, but there are too many
details to give a full treatment in the time we have allocated.  For
those interested in such details, I recommend the 2010 \href{https://www2.eecs.berkeley.edu/Pubs/TechRpts/2010/EECS-2010-37.pdf}{Ph.D.~thesis of
Mark Hoemmen}.

\end{document}
